\documentclass[12pt]{article}
\usepackage[utf8]{inputenc}
\usepackage{style/rsltn}
\usepackage{style/themes/tundra}

\title{An Exercise for Mason}
\author{Robin Truax}
\date{Summer 2023}

\begin{document}

\maketitle

\begin{proposition}[Exercise 1]\label{prop:Exercise_1}
    Suppose that $G$ is a compact, connected topological group, and $U$ is a nonempty open subset of $G$. Then $U^k = G$ for some finite $k$.
\end{proposition}
\begin{proof}
    Consider the smallest subgroup $H$ generated by $U$. Since $H = \bigcup_{n \in \Z}U^n$, and $U^n$ is open for each $n \in \Z$, $H$ is an open subgroup of $G$. Thus, $H$ is a closed subgroup of $G$ whence by connectedness $H = G$. It then follows that $\bigcup_{n \in \Z}U^n$ is an open subcover of $G$, so it has a finite subcover $U^{-n_1} \cup \cdots \cup U^{-n_k} \cup U^{m_1} \cup \cdots \cup U^{m_l}$. Now, both $U^{m_1} \cup \cdots \cup U^{m_l}$ and $U^{-n_1} \cup \cdots \cup U^{-n_k}$ are open, and they cover $U$. Thus, either they are disjoint, or they intersect.\\

    In the former case, $U^{m_1} \cup \cdots \cup U^{m_l}$ is open and closed, and thus is equal to $G$. Thus, in particular, $1 \in U^m$ for some $m$. In the latter case, there exists some $m_i,n_j$ such that $U^{m_i} \cap U^{-n_j}$ has an intersection. It then follows that $1 \in U^{m_i + n_j}$. Again, $1 \in U^m$ for some $m$.\\

    Now, let $W = U^m \cap U^{-m}$; $W$ is symmetric and open, so the subgroup $J$ generated by $W$ is equal to $\bigcup_{k \in \Z^+} W^k$. Since $J$ is the union of open sets, it is open, whence by the same reasoning as earlier $J = G$. Thus, since $G$ is compact, $\{W_k\}_{k=1}^{\infty}$ is an open cover of $G$, and as $W^k$ is increasing in $k$, it follows that $G = W^N$ for some $N$. Since $W \subseteq U^m$, $W^N \subseteq U^{mN}$, whence $U^{mN} = G$, as desired.
\end{proof}
\begin{equation*}
    x^2
\end{equation*}

% \section{Meeting with Katy, July 7th}
% \begin{equation*}
%     \begin{bmatrix}
%         1 & 0 \\
%         0 & -1
%     \end{bmatrix}
%     \begin{bmatrix}
%         0 & 1 \\
%         0 & 0
%     \end{bmatrix}
% \end{equation*}
% \begin{equation*}
% \left[\begin{matrix}
%         e^{1} & 0       \\
%         0     & e^{- 1}
%     \end{matrix}\right]
% \left[\begin{matrix}
%         1 & 1 \\
%         0 & 1
%     \end{matrix}\right]
%     - 
% \left[\begin{matrix}
%         1 & 1 \\
%         0 & 1
%     \end{matrix}\right]
% \left[\begin{matrix}
%         e^{1} & 0       \\
%         0     & e^{- 1}
%     \end{matrix}\right]
%     = \begin{bmatrix}
%     e & e      \\
%     0 & e^{-1}
% \end{bmatrix}
%  - 
% \begin{bmatrix}
%     e & e^{-1} \\
%     0 & e^{-1}
% \end{bmatrix}
% = \begin{bmatrix}
%     0 & e - e^{-1} \\
%     0 & 0
% \end{bmatrix}
% \end{equation*}
%
% \begin{definition}[Primitive Elements]\label{def:Primitive_Elements}
%     Consider a free group $F_n$ with standard basis $x_1, \ldots, x_n$. A \emph{primitive element} is an element $y$ such that there exists a basis $y, y_2, \ldots, y_n$ for $F_n$. 
% \end{definition}
%
% \begin{example}[Examples of Primitive Elements]\label{exam:Examples_of_Primitive_Elements}
%     In the notation given above,\\
%
%     \begin{enumerate}
%         \itemsep1em
%         \item $x_1x_2$ is a primitive element, since $\{x_1x_2, x_2, \ldots, x_n\}$ is a basis for $F_n$.
%         \item $x_1^2$ is not a primitive element. To see why, consider a group homomorphism $\phi: F_n \rightarrow G$. Then, if there existed a basis $\{x_1^2, y_2, \ldots, y_n\}$ for $F_n$, these elements would satisfy the universal property for free groups for $F_n$; that is, for any group $G$ and list $g_1,\ldots,g_n \in G$, there exists a unique homomorphism such that $\phi(x_1^2) = g_1$ and $\phi(y_i) = g_i$. 
%
%             Yet this is not possible, for if $G = \Z_2$, then $\phi(x_1^2) = 2\phi(x_1) = 0$. Thus, the basis fails to satisfy the universal property for the list $g_1 = \cdots = g_n = 1$.
%     \end{enumerate}
% \end{example}
%
% \begin{problem}[Classifying Primitive Elements]\label{prob:Classifying_Primitive_Elements}
%     Let $X = \{a_1, \ldots, a_n\} \subseteq F_n$ and $H = \left<a_1, \ldots, a_n\right> = \left<X\right>$. Does $H$ contain a primitive element?
% \end{problem}
%
% The answer is related to Whitehead's algorithm, which addresses this problem: 
%
% \begin{problem}[Automorphisms on Sets of $\color{rsltnEnvProblem} n$ Words]\label{prob:Automorphisms_on_sets_of_n_words}
%     Suppose that $U = \left\{u_1, \ldots, u_m\right\}$ and $V = \left\{v_1, \ldots, v_m\right\}$ are two sets of words in $F_n$. Then, does there exist $\phi \in \Aut(F_n)$ such that $\phi(U) = V$.
% \end{problem}
%
% Of course, this is trivial for $m = 1$, but is immediately more complicated for $m > 1$.
%
% % \section{Dense Generation of $\SU(n)$ and $\SO(n)$}
% % \section{The Ping-Pong Lemma and Projective Space}
% %
% % \begin{lemma}[Ping-Pong Lemma]\label{lem:the_ping_pong_lemma}
% %     Let $\Gamma$ be a group acting on a set $X$.
% %     Suppose there exist ``rackets" $x,y \in \Gamma$, and nonempty disjoint ``balls" $A,B,C,D \subseteq X$, such that
% %
% %     \begin{align*}
% %         x \cdot (A \cup C \cup D)      & \subseteq A \\
% %         x^{-1} \cdot (B \cup C \cup D) & \subseteq B \\
% %         y \cdot (A \cup B \cup C)      & \subseteq C \\
% %         y^{-1} \cdot (A \cup B \cup D) & \subseteq D
% %     \end{align*}
% %
% %     Then, $\left<x,y\right> = F_2$, the free group on 2 elements.
% % \end{lemma}
% % \begin{proof}
% %     It must only be shown that any nonempty reduced (no consecutive elements are inverses) sequence (word) of rackets is not the identity.
% %     The key idea is that any racket $x,y,x^{-1},y^{-1}$ collapses two of these nonempty balls into a single ball, and any following racket hit \emph{except the first racket's inverse} sends that ball into another, single, ball.
% %     Thus, any nonempty reduced word collapses two of these balls into a single ball, and therefore cannot be the identity.
% %     The \color{rsltnText} result follows, then, from the element-wise definition of the free group.
% % \end{proof}
% %
% % Now, there is a generalization of this result which may prove useful:
% %
% % \begin{lemma}[Ping-Pong Lemma 2]\label{lem:the_ping_pong_lemma_2} Let $\Gamma$ be a group acting on a set $X$.
% %     Suppose that there exist subgroups $G,H \leq \Gamma$, and disjoint nonempty $A,B \subseteq X$ such that \begin{enumerate}
% %         \item $|G| \geq 3$, \item $(G \setminus \{1\})B \subseteq A$, \item $(H \setminus \{1\})A \subseteq B$.
% %     \end{enumerate}
% %     Then, $\left<G,H\right> \simeq G * H$ (the free product of $G$ and $H$).
% % \end{lemma}
% % \begin{proof}
% %     Let $a_1 \cdots a_n$ be a reduced word.
% %     By collapsing any elements in the same group, we may assume that the word is further reduced so that $a_i \in G \Rightarrow a_{i+1} \not\in G$, and $a_i \in H \Rightarrow a_{i+1} \not\in H$ for all $i$.
% %     Then, suppose $a_1,a_n \in G$.
% %     Then, $a_1 \cdots a_n$ sends $B$ to $A$, so it is not the identity.
% %     On the other hand, suppose that at least one of $a_1,a_n \not\in G$.
% %     Then, since $G$ has order at least $3$, there exists an element $a \in G$ distinct from $a_1^{-1}, a_n^{-1}$ and $1$, so that $a^{-1}a_1 \cdots a_na$ is still reduced.
% %     By the previous case, the expanded word is not the identity; thus, it cannot be the case that $a_1 \cdots a_n$ is the identity.
% % \end{proof}
% %
% % \begin{corollary}[Ping-Pong Lemma]\label{cor:the_ping_pong_lemma}
% %     See Lemma \ref{lem:the_ping_pong_lemma}.
% % \end{corollary}
% % \begin{proof}
% %     The result follows because the given assumption immediately implies that both $x$ and $y$ have infinite order; for example, if $x^m = 1$ for any $m > 0$, we get a contradiction by considering the action of each side on $C$ (the right-hand side sends it to $A$, and the left-hand side sends it to $C$).
% %     Then one can choose $G = \left<x\right>$, $H = \left<y\right>$ (where infinite order implies $G$ is large enough), $A' = A \cup B$, and $B' = C \cup D$, and then the result follows from Lemma \ref{lem:the_ping_pong_lemma_2}.
% % \end{proof}
% %
% % \begin{example}[The Special Linear Group]\label{ex:the_special_linear_group}
% %     Consider the action of the special linear group $\SL_2(\R)$ on the projective line $\R\P^1$ given by sending the line $L$ through the origin to the line $AL$ through the origin.
% %     Then, let \begin{equation*} x = \begin{bmatrix}
% %             100 & 0             \\
% %             0   & \frac{1}{100}
% %         \end{bmatrix} \qquad y = R_{\pi/4}\begin{bmatrix}
% %             100 & 0              \\
% %             0   & \frac{1}{100}.
% %         \end{bmatrix}
% %         R_{-\pi/4}.
% %     \end{equation*}
% %     Notice that any line which is not close to the $y$-axis is mapped to a line which is close to the $x$-axis.
% %     Thus, one may define those lines which are close to the $x$-axis to be $A$, and those lines which are close to the $y$-axis to be $B$.
% %     On the other hand, because of the conjugation by a rotation of angle $\frac{\pi}{4}$, a similar result \color{rsltnText} holds for $y$ regarding the lines $x = y$ (lines near which are defined to be $C$) and $x = -y$ (lines near which are defined to be $D$).
% %     The required rules of Lemma \ref{lem:the_ping_pong_lemma} follow, so that $x$ and $y$ generate $F_2$, as desired.
% % \end{example}
% %
% % \subsection{Unipotent Ping-Pong}

\end{document}
