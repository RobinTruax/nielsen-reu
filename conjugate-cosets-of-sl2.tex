\documentclass[12pt]{article}
\usepackage[utf8]{inputenc}
\usepackage[tokyonightDark]{rsltn-style/v3}

\title{Towards Conjugate Cosets of $\SL_2(\C)$ in $\SL_n(\C)$}
\author{Robin}
\date{\today}

\begin{document}
\maketitle
\tableofcontents
\newpage
\thispagestyle{empty}

\section{Lie Groups and Lie Algebras}

\begin{definition}[Lie Group]\label{def:Lie_Group}
    A \emph{Lie group} $G$ is a group and a smooth $n$-manifold so that multiplication $G \times G \rightarrow G$ and $G \rightarrow G$ are both smooth maps.
\end{definition}

\begin{definition}[Homomorphism of Lie Groups]\label{def:Homomorphism_of_Lie_Groups}
    A \emph{homomorphism of Lie groups} $\phi: G \rightarrow H$ is a map which is both differentiable and a group homomorphism.
\end{definition}

\begin{example}[General Linear Group]\label{exam:General_Linear_Group}
    The \emph{general linear group $\GL_n(\C)$} is the set of all invertible $n \times n$ complex matrices under conjugation. It inherits a smooth $n$-manifold structure from $\C^{n \times n}$, of which it is an open subset; multiplication is differentiable as it is a polynomial action, and invertibility is differentiable via Cramer's formula.
\end{example}

\begin{definition}[Representation of a Lie Group]\label{def:Representation_of_a_Lie_Group}
    A \emph{representation} of a Lie group on $V$ is a Lie group homomorphism
    \begin{equation*}
        G \rightarrow \GL(V).
    \end{equation*}
\end{definition}

\begin{example}[Other Examples of Matrix Groups]\label{exam:Other_Examples_of_Matrix_Groups}
    \begin{enumerate}
        \item $\SL_n(\C)$, the set of matrices with determinant 1. 
        \item $B_n(\C)$, the set of upper-triangular matrices.
        \item $N_n(\C)$, the set of upper-triangular unipotent matrices (i.e. 1s on the diagonal).
    \end{enumerate}
\end{example}

\subsection{Properties of Lie Groups}

\begin{proposition}[Generation by Neighborhoods]\label{prop:Generation_by_Neighborhoods}
    Suppose that $G$ is a connected Lie group, and $U \subseteq G$ is any neighborhood of the identity. Then, $U$ generates $G$. 
\end{proposition}
\begin{proof}
    First, notice that by replacing $U$ with $U \cap U^{-1}$, we may assume that $U = U^{-1}$. Let $S$ be the set generated by $U$. Now, of course $S$ is nonempty, and furthermore it is open since any $s \in S$ is contained in the open neighborhood $sU \subseteq S$. Finally, notice that if $s \not\in S$, then the open neighborhood $sU$ is disjoint with $S$. Thus, $S^{\co}$ is open, so that $S$ is closed. Thus, $S$ is nonempty, open, and closed, and in particular equal to $G$. 
\end{proof}

\end{document}
