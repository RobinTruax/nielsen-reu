\documentclass[12pt]{article}
\usepackage[utf8]{inputenc}
\usepackage[tokyonightDark]{rsltn-style/v3}

\title{Lectures with Yizhaq: 1}
\author{Robin}
\date{\today}

\begin{document}
\maketitle

\section{Bases, Hopfian, and Residually Finite Groups}

In these notes, let $F_n$ be the free group of rank $n$. In the case $n = 1$, $F_1 \simeq \Z$. For most questions, we will be interested in the less trivial cases, those for which $n \geq 3$. 

\begin{definition}[Base for Free Group]\label{def:Base_for_Free_Group}
    A \emph{base} for $F_n$ is an $n$-tuple that freely generates $F_n$.
\end{definition}

\begin{theorem}[Redundant Freeness]\label{thm:Redundant_Freeness}
    Any $n$-tuple which generates $F_n$ generates it freely.
\end{theorem}

\begin{definition}[Hopfian]\label{def:Hopfian}
    A group $G$ is said to be \emph{Hopfian} if any surjective homomorphism $\phi: G \rightarrow G$ is also injective, i.e., it is an automorphism. 
\end{definition}

\begin{lemma}[The Free Group is Hopfian]\label{lem:The_Free_Group_is_Hopfian}
    The free group is Hopfian.
\end{lemma}

Using this lemma, Theorem \ref{thm:Redundant_Freeness} follows immediately. 

\begin{proof}[Proof of Proposition \ref{prop:Redundant_Freeness}]
    Suppose that $\{x_1, \ldots, x_n\}$ is a free base for $F_n$ and $\{g_1, \ldots, g_n\}$ is a generating tuple. Then, consider a homomorphism $\phi: F_n \rightarrow F_n$ given by $x_n \rightarrow g_n$. Since $\left<g_1, \ldots, g_n\right> = F_n$, $\phi$ is surjective; since $F_n$ is Hopfian, this establishes that $f$ is injective and thus that $\left\{g_1, \ldots, g_n\right\}$ \emph{freely generate $F_n$}.
\end{proof}

To establish Lemma \ref{lem:The_Free_Group_is_Hopfian}, we use the following.

\begin{definition}[Residually Finite]\label{def:Residually_Finite}
    A group $G$ is residaully finite if for all $g \in G \setminus \{1\}$ there is a finite group $F$ and homomorphism $\phi: G \rightarrow F$ such that $\phi(g) \neq 1 \in F$.
\end{definition}

\begin{lemma}[The Free Group is Residually Finite]\label{lem:The_Free_Group_is_Residually_Finite}
    The free group $F_n$ is residually finite.
\end{lemma}
\begin{proof}
    Let $x_1, \ldots, x_n$ be the standard free basis of $F_n$; then, any nontrivial element of $F_n$ is a word $w$ in these elements. 
    Suppose that $w$ is a nontrivial word. Then, there exists some letter $x_i$ which appears to the power $k \neq 0$. Then, one may define a homomorphism $\phi: F_n \rightarrow \Z/2k\Z$ given by $x_j \mapsto \delta_{ji}$
\end{proof}

\begin{proposition}[Residually Finite is Hopfian]\label{prop:Residaully_Finite_is_Hopfian}
    A residually finite finitely generated group is Hopfian.
\end{proposition}
\begin{proof}
    Let $G$ be a finitely generated residually finite group. Let $\phi: G \twoheadrightarrow G$ be a surjective map and $N = \ker \phi$. Let $a_n$ be the number of subgroups of $G$ of index $n$; since $G$ is finitely generated, $a_n$ is finite for all $n$. If $H \leq G$ is of index $n$, then $\phi^{-1}(H)$ is also of index $n$ in $G$ since $\phi$ is onto. Yet $\phi^{-1}(H) \geq N$. Thus, there are $a_n$ subgroups of index $n$ in $G$ that contain $N$. Thus, all subgroups of finite index contain $N$.\\

    Yet, if $G$ is residually finite, for any $g \in G$ which is not the identity, there exists a finite group $F$ and homomorphism $\psi: G \rightarrow F$ such that $\psi(g) \neq 1$. Then, $[G: \Ker \psi] < \infty$, and $\Ker \psi \not\ni g$, whence $g \not\in N$. Thus, $N = \{1\}$.
\end{proof}

Lemma \ref{lem:The_Free_Group_is_Hopfian} follows as a corollary.

\section{Homomorphisms to Other Groups}

Now, if $\{x_1,\ldots,x_n\}$ is a base for $F_n$, then if $G$ is any group. A map $\phi: F_n \rightarrow G$ is determined by $\phi(x_1),\phi(x_2),\ldots,\phi(x_n)$. Therefore, $\Hom(F_n,G) \simeq G^n$, but since this identification requires a basis, this isomorphism is non-canonical.

\begin{definition}[Primitive Elements]\label{def:Primitive_Elements}
    An element $x \in F_n$ is called \emph{primitive} if it belongs to any base. The set of \emph{all} primitive elements is denoted $P_n \subseteq F_n$.
\end{definition}

\begin{example}[Primitive Elements and Automorphisms]\label{exam:Primitive_Elements_and_Automorphisms}
    Suppose that $x \in F_n$ is primitive. Then the orbit of $x$ under $\Aut(F_n)$, that is, $\Aut(F_n)(x)$, is precisely $P_n$.
\end{example}

\begin{example}[Self-Embeddings]\label{exam:Self_Embeddings}
    Consider the natural embedding $F_n \ideal \Aut(F_n)$ given by taking each element to its corresponding inner automorphism. Then, $P_n \subseteq F_n$ is a conjugacy class in $\Aut(F_n)$.
\end{example}

\begin{problem}[Learn About Primitive Elements]\label{prob:Learn_About_Primitive_Elements}
    Learn about the set of primitive elements $P_n$.
\end{problem}

\end{document}
