\documentclass[12pt]{article}
\usepackage[utf8]{inputenc}
\usepackage[tokyonightDark]{rsltn-style/v3}

\title{Dynamics of Automorphisms of $F_n$}
\author{Roh}
\date{Summer 2023}

\begin{document}

\maketitle
\tableofcontents

\section{Nielsen Moves}
\section{Dense Generation of $\SU(n)$ and $\SO(n)$}
\section{The Ping-Pong Lemma and Projective Space}

\begin{lemma}[Ping-Pong Lemma]\label{lem:the_ping_pong_lemma}
    Let $\Gamma$ be a group acting on a set $X$.
    Suppose there exist ``rackets" $x,y \in \Gamma$, and nonempty disjoint ``balls" $A,B,C,D \subseteq X$, such that

    \begin{align*}
        x \cdot (A \cup C \cup D)      & \subseteq A \\
        x^{-1} \cdot (B \cup C \cup D) & \subseteq B \\
        y \cdot (A \cup B \cup C)      & \subseteq C \\
        y^{-1} \cdot (A \cup B \cup D) & \subseteq D
    \end{align*}

    Then, $\left<x,y\right> = F_2$, the free group on 2 elements.
\end{lemma}
\begin{proof}
    It must only be shown that any nonempty reduced (no consecutive elements are inverses) sequence (word) of rackets is not the identity.
    The key idea is that any racket $x,y,x^{-1},y^{-1}$ collapses two of these nonempty balls into a single ball, and any following racket hit \emph{except the first racket's inverse} sends that ball into another, single, ball.
    Thus, any nonempty reduced word collapses two of these balls into a single ball, and therefore cannot be the identity.
    The \color{rsltnText} result follows, then, from the element-wise definition of the free group.
\end{proof}

Now, there is a generalization of this result which may prove useful:

\begin{lemma}[Ping-Pong Lemma 2]\label{lem:the_ping_pong_lemma_2} Let $\Gamma$ be a group acting on a set $X$.
    Suppose that there exist subgroups $G,H \leq \Gamma$, and disjoint nonempty $A,B \subseteq X$ such that \begin{enumerate}
        \item $|G| \geq 3$, \item $(G \setminus \{1\})B \subseteq A$, \item $(H \setminus \{1\})A \subseteq B$.
    \end{enumerate}
    Then, $\left<G,H\right> \simeq G * H$ (the free product of $G$ and $H$).
\end{lemma}
\begin{proof}
    Let $a_1 \cdots a_n$ be a reduced word.
    By collapsing any elements in the same group, we may assume that the word is further reduced so that $a_i \in G \Rightarrow a_{i+1} \not\in G$, and $a_i \in H \Rightarrow a_{i+1} \not\in H$ for all $i$.
    Then, suppose $a_1,a_n \in G$.
    Then, $a_1 \cdots a_n$ sends $B$ to $A$, so it is not the identity.
    On the other hand, suppose that at least one of $a_1,a_n \not\in G$.
    Then, since $G$ has order at least $3$, there exists an element $a \in G$ distinct from $a_1^{-1}, a_n^{-1}$ and $1$, so that $a^{-1}a_1 \cdots a_na$ is still reduced.
    By the previous case, the expanded word is not the identity; thus, it cannot be the case that $a_1 \cdots a_n$ is the identity.
\end{proof}

\begin{corollary}[Ping-Pong Lemma]\label{cor:the_ping_pong_lemma}
    See Lemma \ref{lem:the_ping_pong_lemma}.
\end{corollary}
\begin{proof}
    The result follows because the given assumption immediately implies that both $x$ and $y$ have infinite order; for example, if $x^m = 1$ for any $m > 0$, we get a contradiction by considering the action of each side on $C$ (the right-hand side sends it to $A$, and the left-hand side sends it to $C$).
    Then one can choose $G = \left<x\right>$, $H = \left<y\right>$ (where infinite order implies $G$ is large enough), $A' = A \cup B$, and $B' = C \cup D$, and then the result follows from Lemma \ref{lem:the_ping_pong_lemma_2}.
\end{proof}

\begin{example}[The Special Linear Group]\label{ex:the_special_linear_group}
    Consider the action of the special linear group $\SL_2(\R)$ on the projective line $\R\P^1$ given by sending the line $L$ through the origin to the line $AL$ through the origin.
    Then, let \begin{equation*} x = \begin{bmatrix}
            100 & 0             \\
            0   & \frac{1}{100}
        \end{bmatrix} \qquad y = R_{\pi/4}\begin{bmatrix}
            100 & 0              \\
            0   & \frac{1}{100}.
        \end{bmatrix}
        R_{-\pi/4}.
    \end{equation*}
    Notice that any line which is not close to the $y$-axis is mapped to a line which is close to the $x$-axis.
    Thus, one may define those lines which are close to the $x$-axis to be $A$, and those lines which are close to the $y$-axis to be $B$.
    On the other hand, because of the conjugation by a rotation of angle $\frac{\pi}{4}$, a similar result \color{rsltnText} holds for $y$ regarding the lines $x = y$ (lines near which are defined to be $C$) and $x = -y$ (lines near which are defined to be $D$).
    The required rules of Lemma \ref{lem:the_ping_pong_lemma} follow, so that $x$ and $y$ generate $F_2$, as desired.
\end{example}

\subsection{Unipotent Ping-Pong}

\end{document}
